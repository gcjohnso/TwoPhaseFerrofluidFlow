\documentclass[11pt,fullpage]{article}
%\usepackage{multicol,wrapfig,amsmath,subfigure}
\usepackage{amsmath}
\usepackage{amsfonts,amssymb}
\usepackage{ mathrsfs }
\usepackage{amsthm}
\usepackage{graphics,graphicx}
\usepackage{hyperref}
\usepackage{ stmaryrd }
\usepackage[right = 2.5cm, left=2.5cm, top = 2.5cm, bottom =2.5cm]{geometry}
\pagestyle{plain}
\def\urlfont{\DeclareFontFamily{OT1}{cmtt}{\hyphenchar\font='057}
	\normalfont\ttfamily \hyphenpenalty=10000}

\newcommand*{\plogo}{\fbox{$\mathcal{PL}$}}
\newcommand{\norm}[1]{\|#1\|}
\newcommand{\innprod}[1]{\langle #1 \rangle}
\newcommand{\Tau}{\mathcal{T}}
\newcommand{\lap}{\Delta}
\newcommand{\problem}[1]{\paragraph{Problem #1}}
%\input macros
\newcommand{\deriv}{\mbox{d}}
\newcommand{\Real}{\mathbb R}
\newcommand{\eps}{\epsilon}
\newcommand{\Complex}{\mathbb C}
\newcommand{\abs}[1]{\left\vert#1\right\vert}
\newcommand{\set}[1]{\left\{#1\right\}}
\newcommand{\subheading}[1]{\noindent \textbf{#1}}
\newcommand{\grad}{\nabla}
\newcommand{\diver}{\textup{div} }
\newcommand{\jump}[1]{[#1]}
\newcommand{\limit}[2]{\lim_{#1 \rightarrow #2}}
\newcommand{\mollify}[1]{ \mathcal{J}_\epsilon #1 }
\newcommand{\conv}[2]{#1 \ast #2}
\newcommand{\D}{D}
\newcommand{\K}{\mathcal{K}}
\newcommand{\C}{\mathcal{C}}
\newcommand{\Torus}{\mathbb{T}}
\newcommand{\Integer}{\mathbb{Z}}
\newcommand{\Naturals}{\mathbb{N}}
\newcommand{\ineqtext}[1]{ ^{\text{\tiny #1}}}
\newcommand{\F}{\mathcal{F}}
\newcommand{\bx}{\mathbf{x}}
\newcommand{\by}{\mathbf{y}}
\newcommand{\intab}{\int_{a}^{b}}
\newcommand{\intom}{\int_{\Omega}}
\newcommand{\Ltwo}{{L^2}}
\newcommand{\Ltwoom}{{L^2(\Omega)}}
\newcommand{\Linf}{{L^\infty}}
\newcommand{\Linfom}{{L^\infty(\Omega)}}
\renewcommand{\i}{{i}}
\newcommand{\ip}{{i+1}}
\newcommand{\ipp}{{i+2}}
\newcommand{\im}{{i-1}}
\newcommand{\imm}{{i-2}}
\renewcommand{\j}{{j}}
\newcommand{\jp}{{j+1}}
\newcommand{\jpp}{{j+2}}
\newcommand{\jm}{{j-1}}
\newcommand{\jmm}{{j-2}}
\newcommand{\n}{{n}}
\newcommand{\np}{{n+1}}
\newcommand{\npp}{{n+2}}
\newcommand{\nm}{{n-1}}
\newcommand{\nmm}{{n-2}}

%Nancy's macros
\newcommand{\reg}[1]{#1^\epsilon}
\newcommand{\Lpr}[1]{L^{#1}(\mathbb{R}^n)}
\newcommand{\Lp}[1]{L^{#1}(\Omega)}
\newcommand{\intreal}[1]{\int_{\mathbb{R}^{#1}}\hspace{-8pt}}
\newcommand{\modenergy}{\mathcal{E}_H}
\newcommand{\ball}{B_n}
\newcommand{\balltime}{B_n\times [0,T]}

\newtheorem{theorem}{Theorem}
\theoremstyle{lemma}
\newtheorem{prop}{Proposition}
\newtheorem{proposition}{Proposition}
\newtheorem{corollary}{Corollary}
\theoremstyle{definition}
\newtheorem{definition}{Definition}
\newtheorem{exercise}{Exercise}
\newtheorem{algorithm}{Algorithm}
\newtheorem{question}{Question}
%\theoremstyle{remark}
\newtheorem{remark}{Remark}
\newtheorem{rmk}{Remark}
\newtheorem{example}{Example}
\theoremstyle{lemma}
\newtheorem{lemma}{Lemma}
\allowdisplaybreaks

\begin{document}
	
\begin{center}
	\textbf{\Large{Scientific Computation of Two--Phase Ferrofluid Flows}}
	\\
	\textbf{\Large{AMSC 663/664}} \\ 
	\textbf{\large{Gareth Johnson}}\\
	\textbf{\large{Faculty Adviser: Ricardo Nochetto}}
	\\
	\today
\end{center}

\section{Introduction}
A ferrofluid is a liquid which becomes magnetized when under the effect of a magnetic field. A ferrofluid is a colloid of nanoscale ferromagnetic particles suspended in a carrier fluid such as oil, water, or an organic solvent. Due to their ability to be controlled by external magnetic fields, ferrofluids are used in a wide application of control based problems. Ferrofluids were first used to pump rocket fuel once a spacecraft entered a weightless environment \cite{Rocket}. Commercial uses of ferrofluids include applications in vibration damping, sensors, and acoustics \cite{CommercialAppl}. A more recent area of research is magnetic drug targeting, where drugs are injected and guided using magnetic fields to their specific destination \cite{DrugTarg:1, DrugTarg:2, DrugTarg:3}. A final application worth mentioning is the construction of adaptive deformable mirrors, which can be viewed as a shape optimization problem \cite{FerroMirror:1, FerroMirror:2, FerroMirror:3}. 

For one phase ferrofluid flows, there are two primary PDE models which mathematically describe the behavior of a ferrofluid under the effects of a magnetic field referred to by the names of their creators; the Rosensweig model \cite{Rosensweig} and the Shliomis model \cite{Shliomis}. Both models are current research areas, with existence of global weak solutions and local existence of strong solutions being recent results \cite{PDEResults:1,PDEResults:2, PDEResults:3, PDEResults:4}. However, these models do not describe two--phase ferrofluid flows, where one phase has magnetic properties and the other does not. While work has been done to create interface conditions for two--phase flows in the sharp interface regime \cite{SharpInter:1, SharpInter:2}, there is not an established PDE model which describes two--phase ferrofluid flows.

Using numerical analysis and scientific computation, new models can be developed through trial and error. To this extent, various computational methods have been used in order to numerically simulate various phenomenon of two--phase ferrofluid flows. For stationary phenomena, Tobiska and collaborators have devised numerical and physical experiments investigating the free surface of ferrofluids using a sharp interface approach \cite{Tob:1, Tob:2, Tob:3}. In \cite{Numerics:1, Numerics:2}, Volume of Fluid methods are used to numerically investigate non--stationary phenomena, such as the field induced motion of a ferrofluid droplet and the formation process of ferrofluid droplets respectively. 

An important issue with these techniques is that their numerical implementations, stability, and convergence are not explored. To this end, Nochetto and collaborators developed a model for two--phase ferrofluid flow and devised an energy stable numerical scheme \cite{DiffuseInterface}. The model presented was not derived but instead was assembled by selecting specific components from existing models and standard assumptions. This was done to create a simple model, focusing on a minimal number of constitutive parameters and coupled PDEs, that still captured the basic phenomena of two--phase ferrofluids. Below, we present the model in full and then summarize the derivation of each component and define all variables and parameters.

The model considers a two--fluid system, consisting of a ferrofluid and a non--ferromagnetic one, on a bounded convex polygon/polyhedron domain $\Omega\subset \Real^d$ ($d=2$ or $3$) with boundary $\Gamma$. The evolution of the system is given by the following set of equations in strong form in $\Omega$
\begin{subequations}\label{Model}
	\begin{align}
	\label{Cahn-Hilliard:1}\theta_t + \diver(\mathbf{u}\theta) + \gamma \lap \psi &= 0,\\
	\label{Cahn-Hilliard:2}\psi - \eps \lap \theta + \frac{1}{\eps}f(\theta) &= 0,\\
	\label{Advection-Reaction}\mathbf{m}_t + (\mathbf{u}\cdot \grad)\mathbf{m}&=-\frac{1}{\mathscr{T}}(\mathbf{m} - \varkappa_\theta\mathbf{h}),\\
	\label{MagScalarPot}-\lap \varphi &=\diver(\mathbf{m}-\mathbf{h}_a),\\
	\label{NavierStokes}\mathbf{u}_t + (\mathbf{u}\cdot\grad)\mathbf{u} - \diver (\nu_\theta \mathbf{T(u)}) + \grad p &= \mu_0(\mathbf{m}\cdot \grad)\mathbf{h} + \frac{\lambda}{\eps}\theta\grad \psi,\\
	\label{DivFree}\diver \mathbf{u} &= 0,
	\end{align}
\end{subequations}
for every $t\in[0,t_F]$, where $\mathbf{T(u)}=\frac{1}{2}(\grad \mathbf{u} + \grad\mathbf{u}^T)$ denotes the symmetric gradient and $\mathbf{h}=\grad \varphi$. The system (\ref{Model}) is supplemented with the boundary conditions
\begin{equation}\label{ModelBC}
	\partial_\eta \theta = \partial_\eta\psi = 0,\quad \mathbf{u}=0,\quad\text{and}\quad \partial_\eta\varphi = (\mathbf{h}_a - \mathbf{m})\cdot\eta \quad \text{on }\Gamma.
\end{equation}

The position of each fluid is tracked using a phase variable $\theta$, which makes the model a diffuse--interface type. The phase variable $\theta$ takes values in $[-1,1]$, where $\theta = \pm 1$ denotes a pure concentration of a single fluid and values in $(-1,1)$ denotes the interface between the fluids. The evolution of the phase variable is governed by the Cahn--Hilliard equation, given by (\ref{Cahn-Hilliard:1}) and (\ref{Cahn-Hilliard:2}), where $0< \eps <<1$ is the interface thickness, $\gamma>0$ is the (constant) mobility, $\psi$ is the chemical potential, $f(\theta)=F'(\theta)$ and $F(\theta)$ is the truncated double well potential
\begin{equation}\label{DoublePotent}
	F(\theta) = \begin{cases}
	(\theta + 1)^2 &\text{if }\theta\in(-\infty,-1]\\
	\frac{1}{4}(\theta^2 - 1)^2 &\text{if }\theta\in[-1,1]\\
	(\theta -1)^2 &\text{if }\theta\in[1,+\infty).
	\end{cases}
\end{equation}

The evolution of the magnetization $\mathbf{m}$, which is induced by a magnetic field $\mathbf{h}$ defined as the gradient of a magnetic potential $\varphi$, is given by the simplified advection--reaction equation (\ref{Advection-Reaction}). In (\ref{Advection-Reaction}), $\mathscr{T}$ is the relaxation time of the ferrofluid and $\varkappa_\theta$ is the magnetic susceptibility of the phase variable. Defining $\varkappa_0>0$ to be the magnetic susceptibility of the ferrofluid and setting the non--magnetic fluid to have zero magnetic susceptibility, we have that $\varkappa_\theta$ is a Lipschitz continuous function of $\theta$ satisfying $0\leq \varkappa_\theta \leq \varkappa_0$. The magnetic field $\mathbf{h}$ is the sum of a smooth harmonic applied magnetizing field $\mathbf{h}_a$ (i.e. $\diver \mathbf{h}_a = 0,\quad \text{curl}\mathbf{h}_a=0$) and a de--magnetizing field $\mathbf{h}_d$. It is modeled using a scalar potential $\varphi$ which satisfies $\mathbf{h} = \grad \varphi$ and equation (\ref{MagScalarPot}). 

Finally, the velocity--pressure pair $(\mathbf{u}, p)$ are given by a simplified Navier--Stokes equation (\ref{NavierStokes}) coupled with an incompressibility condition (\ref{DivFree}). In (\ref{NavierStokes}), $\nu_\theta$ is the viscosity of the phase variable, $\mu_0$ is the constitutive parameter related to the Kelvin force, and $\frac{\lambda}{\eps}\theta\grad \psi$ is the capillary force. Defining $\nu_w, \nu_f$ to be the viscosities of the non--magnetic fluid and the ferrofluid respectively, we have that $\nu_\theta$ is a Lipschitz continuous function of $\theta$ satisfying
$$
	0 < \min\set{\nu_w, \nu_f} \leq \nu_\theta \leq \max\set{\nu_w, \nu_f}.
$$

In order to solve system (\ref{Model}), the following numerical scheme is proposed in \cite{DiffuseInterface}. Define $K>0$ to be the number of time steps, with uniform time step $\tau = T/K>0$. Define the backwards difference operator $\delta$:
$$
	\delta f^k = f^k - f^{k-1}.
$$
Define the following finite dimensional subspaces $\mathbb{G}_h\subset H^1(\Omega), \mathbb{Y}_h\subset H^1(\Omega), \mathbb{M}_h\subset L^2(\Omega), \mathbb{X}_h\subset H^1(\Omega), \mathbb{U}_h\subset H^1_0(\Omega),$ and $\mathbb{P}_h\subset L^2(\Omega)$ that will approximate the phase field, chemical potential, magnetization, magnetic potential, velocity, and pressure respectively. In the context of finite elements, the above spaces are parameterized by the meshsize $h$. The pair of spaces $(\mathbb{U}_h, \mathbb{P}_h)$ are assumed to satisfy a uniform inf--sup condition
\begin{equation}
\inf_{0\neq Q\in\mathbb{P}_h}\sup_{0\neq \mathbf{V}\in\mathbb{V}_h}\frac{(\diver \mathbf{V}, Q)}{\norm{Q}_{\Ltwo}\norm{\mathbf{V}}_\Ltwo}\geq \beta^*,
\end{equation}
with $\beta^*>0$ independent of $h$. Introduce a suitable discretization of the trilinear form for the convective term in the Navier--Stokes equation:
$$
\mathcal{B}_h(\cdot,\cdot,\cdot):\mathbb{U}_h\times\mathbb{U}_h\times\mathbb{U}_h\to\Real.
$$
Additionally, introduce a suitable discretization of the trilinear form for the convective term of (\ref{Advection-Reaction}) and the Kelvin force in (\ref{NavierStokes}):
$$
\mathcal{B}_h^m(\cdot,\cdot,\cdot):\mathbf{U}_h\times\mathbf{M}_h\times\mathbf{M}_h\to\Real.
$$
For given smooth initial data $\set{\Theta^0, \mathbf{M}^0,\mathbf{U}^0}$, compute $\set{\Theta^k, \Psi^k, \mathbf{M}^k, \Phi^k, \mathbf{U}^k, {P}^k}\in \mathbb{G}_h\times \mathbb{Y}_h\times \mathbb{M}_h\times\mathbb{X}_h\times\mathbb{U}_h\times\mathbb{P}_h$ for every $k\in\set{1,...,K}$ that solves
\begin{subequations}\label{NumScheme}
	\begin{align}
	\label{NumScheme:1}\bigg(\frac{\delta\Theta^k}{\tau}, \Lambda\bigg) - (\mathbf{U}^k\Theta^{k-1}, \grad\Lambda) - \gamma(\grad \Psi^k, \grad \Lambda)&=0,\\
	\label{NumScheme:2}(\Psi^k, \Upsilon) + \eps(\grad \Theta^k, \grad \Upsilon) + \frac{1}{\eps}(f(\Theta^{k-1}), \Upsilon) + \frac{1}{\eta}(\delta\Theta^k, \Upsilon) &= 0,\\
	\label{NumScheme:3}\bigg(\frac{\delta\mathbf{M}^k}{\tau}, \mathbf{Z}\bigg) - \mathcal{B}_h^m(\mathbf{U}^k, \mathbf{Z}, \mathbf{M}^k) + \frac{1}{\mathscr{T}}(\mathbf{M}^k, \mathbf{Z}) &= \frac{1}{\mathscr{T}}(\varkappa_\theta\mathbf{H}^k, \mathbf{Z}),\\
	\label{NumScheme:4}(\grad\Phi^k, \grad X) &= (\mathbf{h}_a^k - \mathbf{M}^k, \grad X),\\
	\begin{split}
	\label{NumScheme:5}\bigg(\frac{\delta\mathbf{U}^k}{\tau}, \mathbf{V}\bigg) + \mathcal{B}_h(\mathbf{U}^{k-1}, \mathbf{U}^k, \mathbf{V}) + (\nu_\theta\mathbf{T(U}^k), \mathbf{T(V)}) - (P^k, \diver \mathbf{V}) &= \mu_0\mathcal{B}_h^m(\mathbf{V}, \mathbf{H}^k, \mathbf{M}^k)\\
	&\phantom{{}={}}+\frac{\lambda}{\eps}(\Theta^{k-1}\grad \Psi^k, \mathbf{V}),
	\end{split}\\
	\label{NumScheme:6}(Q, \diver\mathbf{U}^k) &= 0,
	\end{align}
\end{subequations}
for all $\set{\Lambda, \Upsilon, \mathbf{Z}, X, \mathbf{V}, Q}\in \mathbb{G}_h\times \mathbb{Y}_h\times \mathbb{M}_h\times\mathbb{X}_h\times\mathbb{U}_h\times\mathbb{P}_h$, where $\mathbf{H}^k=\grad \Phi^k$ and $\eta \leq \big(\max_\theta f'(\theta)\big)^{-1}$. The numerical scheme (\ref{NumScheme}) was proven to be energy--stable and locally solvable \cite{DiffuseInterface}. 


\section{Project Goals}
The primary goals of the project are as follows:
\begin{itemize}
	\item[1)] Finite Element Code: Develop a finite element code to solve two--phase ferrofluid flows using the numerical scheme (\ref{NumScheme}). This code will be written in a ``dimensional--less" way, this is explained in Section 5, so that the code can be easily transitioned from 2d to 3d. This would enable others, after making the required modifications as detailed in the user guide, to perform their own numerical simulations. We note that in order to run the code in 3d, many additional components will have to be implemented that are beyond the scope of what can be done in a year. Thus, the focus will be to get the code to work for 2d simulations.
	
	\item[2)] Solvers: In order to solve (\ref{NumScheme}), three different solvers are required. In Section 4, we present current ideas on which solvers will be used for each system but it may become necessary to investigate other methods for a given system.
	
	\item[3)] Scientific Questions: Investigate the structure of the velocity field of the fluid flow and the magnetic field around the spike deformation of the ferrofluid for both the Rosenswieg instability and the ferrofluid hedgehog configuration in 2d, which are explained in Section 8.
\end{itemize}
If time permits, the following extensions of the project will be explored:
\begin{itemize}
	\item[4)] Parallelization: Implement parallel adaptive mesh refinement/coarsening. The parallelization will involve using multiple cores on a single processor, as opposed to a distributed set of processors, to refine the mesh within a particular partition of the domain.
	
	\item[5)] Ferrofluid Droplets: Investigate if the model can accurately capture various effects of ferrofluid droplets. Would involve comparing numerical results to other computational work such as \cite{CompDropplet}, which investigated the coalescence of droplets using a finite volume based approach, and \cite{DroppletDeform} which investigated the equilibrium shape of droplets under a uniform magnetic field. The main issues with the comparison of results would be that \cite{CompDropplet, DroppletDeform} use different models for the ferrofluid and numerical methods.
\end{itemize}


\section{Approach}
In the first semester the following modules/components have been implemented, and are discussed in more detail in section 6:
\begin{itemize}
	\item[1)] Write code to handle the generation and adaptive refinement/coarsening of the mesh.
	
	\item[2)] Write a code to handle the generation of the magnetic potential, given the locations of each magnetic dipole.
\end{itemize}

\noindent In the second semester the following modules/components will be implemented:
\begin{itemize}
	\item[1)] Write codes to handle the generation of the finite element spaces given in \cite{DiffuseInterface}. The elements used will be polynomials of degree $2$ in each variable.
	
	\item[2)] Write a code to handle the generation of the matrices, which will combine information from the mesh and the finite elements. 
	
	\item[3)] Write codes that solve each of the three subsystems, namely the Cahn--Hilliard system (\ref{NumScheme:1})--(\ref{NumScheme:2}), the magnetization system (\ref{NumScheme:3})--(\ref{NumScheme:4}), and the Navier--Stokes system (\ref{NumScheme:5})--(\ref{NumScheme:6}).
	
	\item[4)] Write a code to solve the full system (\ref{NumScheme}) at each time step using a Picard--like iteration.
	
	\item[5)] Include functionality for the numerical simulation to be restarted from the last completed iteration.
\end{itemize}
The numerical investigation into the structure of the velocity and magnetic field will be performed by analyzing various plots produced from the results for both the Rosenswieg instability and the ferrofluid hedgehog configuration. Specifically we will examine both contour and vector plots in order to better understand these fields.


\section{Numerical Implementation}
The numerical scheme (\ref{NumScheme}) is discretized in time using backward Euler and in space using the Galerkin method. Specifically, equation (\ref{NumScheme:3}) will use discontinuous Galerkin and the remainder will use continuous Galerkin. The choice to use discontinuous Galerkin is due to the natural upwinding it provides for the mass dominated system (\ref{NumScheme:3}). Backward Euler was chosen because we expect the velocity of the solution to be rather small and it allows us to use a larger time--step without any restriction on the meshsize. In order to solve the scheme, a Picard--like iteration is used. Specifically, each iteration is comprised of repeatedly solving three subsystems until the velocity has converged to a fixed point. The choice to use a Picard--like iteration was because other attempts to further uncouple the system yielded unsatisfactory results and in practice the Picard--like scheme appeared to converge to a fixed point for this particular problem \cite{DiffuseInterface}. The process at the $k$--th iteration is as follows: 

Given $\mathbf{U}^{k-1}$ 
\begin{itemize}
	\item[1)] Compute $\Theta^k$ and $\Psi^k$ substituting $\mathbf{U}^{k-1}$ for $\mathbf{U}^{k}$.
	
	\item[2)] Next compute $\mathbf{M}^k$ and $\Phi^k$ using $(\Theta^k,\Psi^k)$ from the previous iteration and substituting $\mathbf{U}^{k-1}$ for $\mathbf{U}^{k}$.
	
	\item[3)] Finally, compute $\mathbf{U}^k$ and $P^k$ using $(\Theta^k,\Psi^k, \mathbf{M}^k, \Phi^k)$ from the previous two iterations.
	
	\item[4)] Repeat steps 1-3 using $\mathbf{U}^k$ from the previous iteration as input until $\mathbf{U}^k$ does not change between iterations.
\end{itemize}


As detailed in Section 2, various solvers are needed. First, the Cahn--Hilliard equations (\ref{NumScheme:1})--(\ref{NumScheme:2}) are solved by utilizing convex--concave splitting thus resulting in a linear algebraic system without constraints on the time-step and regularization parameter \cite{CahnHilliard}. Due to the resulting matrix being non--symmetric, due to the second term in (\ref{NumScheme:1}), GMRES preconditioned with algebraic multigrid will be used. Next the magnetization problem (\ref{NumScheme:3})--(\ref{NumScheme:4}) will use a different solver for each equation. As mentioned previously, (\ref{NumScheme:3}) is a mass dominated system which is also non--symmetric and thus BiCGstab will be used. (\ref{NumScheme:4}) is just a Laplacian, thus we will use CG preconditioned with algebraic multigrid. Finally, the Navier--Stokes equations (\ref{NumScheme:5})--(\ref{NumScheme:6}) is a non--symmetric saddle point problem, thus it will be solved using GMRES. The saddle point system will need to be preconditioned using a block preconditioner, we will explore preconditioners presented in Elman's book \cite{Precond}.

The generation of the applied magnetizing field is given by the following procedure: Define the magnetic dipole $\phi_s$ by
$$
\phi_s(\mathbf{x}) = \frac{\mathbf{d}\cdot(\mathbf{x}_s - \mathbf{x})}{\abs{\mathbf{x}_s - \mathbf{x}}}, 
$$
where $\abs{\mathbf{d}}=1$ and $\mathbf{d},\mathbf{x}_s,\mathbf{x}\in\Real^2$. Then the applied magnetizing field $\mathbf{h}_a$ is given by
\begin{equation}\label{AMF}
\mathbf{h}_a = \sum_s \alpha_s(t) \grad\phi_s,
\end{equation}
where $\alpha_s(t)$ is the intensity of each dipole. The code will support both constant and linearly increasing intensities.

The adaptive mesh refinement/coarsening will use the simplest element indicator $\eta_T$ \cite{ErrorInd}:
\begin{equation}\label{ErrorInd}
	\eta_T^2 = h_T\int_{\partial T}\abs{\bigg\llbracket\frac{\partial \Theta}{\partial \eta}\bigg\rrbracket}^2dS \quad \forall T\in\Tau_h.
\end{equation}


\section{Hardware and Software Platform}
We are still targeting a Linux desktop system. Since our focus is to get the code working for 2d problems without parallel adaptive mesh refinement/coarsening, a multi--code processor will not be necessary. The amount of RAM necessary to run the simulations described in Section 8, even in 2d, may be quite large. For these reasons, the code will be tested on a desktop owned by Dr. Nochetto, which has a Intel Xeon CPU with 24 cores and 68 GB of RAM.

The code is being written in C++. This decision was made in order to utilize the deal.II library \cite{AdaptiveMesh:1,DealII}. deal.II has functionality to create meshes, generate finite elements, handle adaptive mesh coarsening and refinement (including implementation of the error indicator (\ref{ErrorInd})), and contains a variety of linear algebra solvers and preconditioners. As mentioned earlier, deal.II also offers a method of parameterizing the dimension of your problem allowing the ability to transition between 2d and 3d without having to rewrite the entire code.

\section{Project Progress}
This semester has been focused on overcoming the learning curve of solving PDEs using the deal.II library. While the library provides many tools to aid in developing a finite element code, these tools are complex and need to be used in specific ways in order to write efficient code. In order to acclimate myself with the library, I spent most of the semester following along and implementing various programs described in the deal.II tutorial \cite{DealIITut}. I picked specific tutorials which used and explained functionality that would be needed by my own code. Below is a list summarizing the functionality that I have learned throughout the various tutorials:
\begin{itemize}
	\item[1)] Handling of time dependent problems. This includes the implementation of the time discretization and the procedure for adaptively refining/coarsening the mesh using the Kelly error estimator (\ref{ErrorInd}).
	
	\item[2)] Handling the construction of the various matrices corresponding to the bilinear forms present in (\ref{NumScheme}) for $Q_2$ elements.
	
	\item[3)] A method for solving a saddle point system using the Schur complement technique.
	
	\item[4)] Interfacing with various numerical linear algebra solvers and how to apply preconditioners to the particular solver. Currently, I have interfaced with CG but the process of changing to BiCGstab or GMRES is minimal.
	
	\item[5)] A method for computing various errors of the numerical solution when the exact solution is known, which will be used in verification.
	
	\item[6)] How to use discontinuous elements in deal.II and how to compute the bilinear forms which correspond to integrals over the faces of elements.
\end{itemize}
For a more detailed description of each tutorial and the specific functionality learned please refer to the document titled "deal.II Example Code Documentation" which can be found in the Example folder of the Github repository. Currently, I am investigating how to use algebraic multigrid as a preconditioner for GMRES, which is discussed in the tutorial step--31. Once this is done, I will have covered all of the preliminary functionality of deal.II and will be ready to begin implementation of the three solvers for (\ref{NumScheme}).

In addition to the above, I have also completed and unit tested two of the modules/components as described in Section 3. First, the code that generates the initial mesh for the experiments described in Section 8 has been implemented and unit tested. In addition, the code for adaptively refining/coarsening the mesh was used multiple times in the various tutorial programs. Specifically, the time dependent version implemented in step--26 will be used with very little modification. Second, the code to generate the magnetic dipole has been implemented and unit tested. The structure of the configuration file which describes the intensity and location of each dipole has also been documented.

\section{Documentation and Distribution}
The project is being distributed using the public Github repository found at \url{https://github.com/gcjohnso/TwoPhaseFerrofluidFlow}. Github was chosen as it will provide a central location for the code and documentation to be located.

The project is being documented in two ways. First, a user guide is being developed on the projects Github page. The user guide is being updated as new functionality is implemented. Currently, the documentation describes mesh generation and the generation of the applied magnetizing field. Second, is general code comments, including but not limited to: file, class, and method level block comments. 

The user guide will include documentation of the model and the numerical scheme (including its implementation), the architecture of the code, and instructions on the components that will need to be modified in order for users to run their own simulations. By including a description about how the various components interact, the user will hopefully have a better idea of how to extend the code to run simulations currently out of scope for the current design.


\section{Validation Methods}
The following methods will be used to validate the code:
\begin{itemize}
	\item[1)] Generated solutions: A technique for generating a specific analytic solution is to add a forcing function which will make the equation hold for the given analytic solution. This technique will be used to verify that each of the individual solvers for the coupled equations (\ref{NumScheme:1})--(\ref{NumScheme:2}), (\ref{NumScheme:3})--(\ref{NumScheme:4}), and (\ref{NumScheme:5})--(\ref{NumScheme:6}) are working as intended.
	
	\item[2)] Mesh verification: First, the input mesh will be visually verified by looking at the mesh output generated by deal.II.
	
	\item[3)] Applied Magnetizing Field verification: The code used to return the value of the applied magnetizing field given a set of dipoles and intensities is currently passing all unit tests. This was done by comparing the computed answer against the exact answer for simple dipole configurations.
	
	\item[4)] Comparison with prior work: Three numerical simulations were carried out in \cite{DiffuseInterface}. The first experiment reproduced the Rosensweig instability, which occurs when a horizontal pool of ferrofluid is under the combined effects of gravity and a uniform magnetic field pointing upwards. Under these conditions the flat profile of the ferrofluid pool is no longer stable, instead favoring a pattern of peaks and valleys. The second and third experiments focused on what the authors refereed to as the ferrofluid hedgehog. The ferrofluid hedgehog was a phenomena exhibited when a pool of ferrofluid was acted on by gravity and a non--uniform magnetic field. They showed numerically that the de--magnetizing field was necessary in order for the ferrofluid hedgehog to form, as was the case in experiment two which utilized the complete magnetizing field. However, in experiment three, which used a reduced magnetizing field, leaving out the de--magnetizing field, the ferrofluid instead formed a oval shape which does not resemble what would happen in a real world experiment.
	
	In order to validate the whole code, I will recreate the results described above. This is possible since detailed parameters were given in \cite{DiffuseInterface} for each experiment.
\end{itemize}


\section{Deliverables}
The following are the current deliverables for the project and are available in the Github repository:
\begin{itemize}
	\item[1)] The source code and documentation for the deal.II example code discussed in Section 6.
	
	\item[2)] The source code, validation tests, and documentation of the mesh and applied magnetizing field components.
\end{itemize}

The following will be the final deliverables for each of the project goals given at the end of next semester:
\begin{itemize}
	\item[1)] All source code, validation tests, and documentation for the two--phase ferrofluid flow solver will be available on the Github page. Figures and video of the three numerical simulations used for verification will be created and delivered to the teaching staff. 
	
	\item[2)] The solvers will be included in the source code and a description of each will be provided in the documentation along with a justification to why each was used.
	
	\item[3)] A report will be written detailing the findings of the numerical investigations into the structure of the velocity and magnetic field for the Rosenswieg instability and the ferrofluid hedgehog configuration.
\end{itemize}


\section{Timeline}
Below is a updated timeline for the remainder of the project:
\begin{itemize}
	\item Mid January: Begun development on the solver for the Cahn--Hilliard system (\ref{NumScheme:1})--(\ref{NumScheme:2}).
	
	\item Mid February: Finished and tested the solver for the Cahn--Hilliard system (\ref{NumScheme:1})--(\ref{NumScheme:2}). Begun development on the solver for the Navier--Stokes system (\ref{NumScheme:5})--(\ref{NumScheme:6}).
	
	\item Mid March: Finished and tested the solver for the Navier--Stokes system (\ref{NumScheme:5})--(\ref{NumScheme:6}). Begun development on the solver for the magnetization system (\ref{NumScheme:3})--(\ref{NumScheme:4}). 
	
	\item Mid April: Finished and tested the solver for the magnetization system (\ref{NumScheme:3})--(\ref{NumScheme:4}). Begun development on the Picard iteration solver.
	
	\item Semester End: Delivered all required items and the final report to the teaching staff.
\end{itemize}


	
\bibliographystyle{siam}
\bibliography{Final}
\end{document}
